\documentclass[10pt,a4paper]{article}
\linespread{1.2}
\usepackage{verbatim}
\usepackage{geometry}
\usepackage{listings}

\geometry{right=2.0cm,left=2.0cm,top = 2.0cm, bottom = 2.0cm}

\lstdefinestyle{mystyle}{
    basicstyle=\ttfamily
}

\lstset{style=mystyle}

\title{CAAM 519, Homework \#1: \LaTeX\  Submission}
\author{\texttt{hc54}}
\date{September 7, 2022}

\begin{document}

\maketitle

\section{Communicating with remote machines via ssh}
\subsection{Ssh command used to log into the clear machine}
\begin{verbatim}
henry@henry-VirtualBox:~/caam419/caam-419-519-submissions/homework-1$ ssh hc54@ssh.clear.rice.edu
\end{verbatim}

\subsection{The message that the clear machine provided upon logging in}
\begin{verbatim}
The Rice University Network - Unauthorized access is prohibited
(hc54@ssh.clear.rice.edu) Password: 
(hc54@ssh.clear.rice.edu) Duo two-factor login for hc54

Enter a passcode or select one of the following options:

 1. Duo Push to +XX XXX XXXX 3987
 2. Phone call to +XX XXX XXXX 3987
 3. SMS passcodes to +XX XXX XXXX 3987

Passcode or option (1-3): 1
Success. Logging you in...
The Rice University Network
 ===========================
 Unauthorized use is prohibited.
 
 This computer system is for authorized users only.  Individuals using this
 system without authority or in excess of their authority are subject to
 having all their activities on this system monitored and recorded or
 examined by any authorized person, including law enforcement, as system
 personnel deem appropriate.  In the course of monitoring individuals
 improperly using the system or in the course of system maintenance, the
 activities of authorized users may also be monitored and recorded.  Any
 material so recorded may be disclosed as appropriate.  Anyone using this
 system consents to these terms.
 
 Problems and/or questions should be submitted via the problem tracking
 system form: http://helpdesk.rice.edu
 
CURRENT USAGE AND LOAD ON THE COMPUTE NODES:
  Sun Sep 18 17:40:01 CDT 2022

 System                   	# Users   	Load ( 5, 10, 15 minute)      
   agate.clear.rice.edu   	    2     	  0.00, 0.01, 0.05            
   amber.clear.rice.edu   	    0     	  2.84, 2.78, 2.83            
   cobalt.clear.rice.edu  	    0     	  0.01, 0.02, 0.05            
   jade.clear.rice.edu    	    1     	  0.05, 0.07, 0.05            
   onyx.clear.rice.edu    	    3     	  0.07, 0.05, 0.12            
   opal.clear.rice.edu    	    2     	  5.49, 5.50, 5.37            
   pyrite.clear.rice.edu  	    1     	  0.00, 0.03, 0.05            

NOTE: !!!!!!!!!!!!!!!!!!!!!!!!!!!!!!!!!!!!!!!!!!!!!!!!!!!!!!!!!!!!!!!!!!!
 
   Please log an RT ticket for any issues you may have.
 
   Please log a ticket at https://help.rice.edu if you have any of the following:
     Feel documentation is lacking.
     Have trouble getting into the system.
     Feel the system is missing a tool.
 
   NOTE: If you don't have a home drive when using Clear, please 
         create a ticket in help.rice.edu  Be sure to put 
         "need clear home directory" in the subject and one will 
         be setup for you. Include your Course (dept and number) or 
         a faculty sponsor if not for a course.
 
   NOTE: If you see "id: cannot find name for group ID <Number>" at 
         login, you can safely ignore this and move on with your work.
 
   FACULTY: Please help us pre-populate home drives for your class.
         Provide us with your course number in an RT ticket like Note above.
 
   NOTE: rclone added to clear. More information on kb.rice.edu soon.
 
   NOTE: ghc (haskell) updated on clear. Run "PATH=$PATH:/clear/apps/ghc/bin" to use.
 
   CLEAR NEWS -- https://kb.rice.edu/internal/page.php?id=71856
   Tips and Hints -- https://kb.rice.edu/internal/page.php?id=71857
 

Could not chdir to home directory /storage-home/h/hc54: No such file or directory    
\end{verbatim}

\subsection{The output of the echo}
\begin{verbatim}
[hc54@onyx /]$ echo $HOSTNAME
onyx.clear.rice.edu    
\end{verbatim}

\section{A script to build a LaTeX document while hiding auxiliary files}

\begin{lstlisting}[language=bash,caption={Shell Script for hiding auxiliary files}]
#!/bin/bash
mkdir -p .build
read project
pdflatex $project.tex
mv $(ls | grep -v $project.pdf) ./.build
mv ./.build/clean-build.sh ./
\end{lstlisting}

The first line is to tell the computer that this is a shell script. The second line is to create a hidden directory called .build in current path only when current path does not have a directory with the same name. The third line is to tell the computer that this shell will take one input. The fourth line is to generate one pdf file given the project name. The fifth line is to move all files except the pdf file to the hidden build directory. The last line is to move the shell script back.

\noindent The first unexpected outcome is that this shell script will fail to generate more than one pdf file. If there are more than one tex file in the directory initially, running the shell for first project will move all other files, including the original tex files of other projects, into the hidden directory. Another unexpected behavior is that when there are multiple tex files in the same directory, the shell will store all the auxiliary files in one single hidden directory, The user might want to store the auxiliary files like log and bibliography of different projects into different hidden directories. 

\end{document}
